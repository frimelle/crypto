\documentclass[11pt]{article}

\usepackage{listings}
\usepackage{amsmath}
\usepackage{todonotes}
\usepackage{hyperref}

\usepackage{color}

\definecolor{dkgreen}{rgb}{0,0.6,0}
\definecolor{gray}{rgb}{0.5,0.5,0.5}
\definecolor{mauve}{rgb}{0.58,0,0.82}

\lstset{frame=tb,
  language=Java,
  aboveskip=3mm,
  belowskip=3mm,
  showstringspaces=false,
  columns=flexible,
  basicstyle={\small\ttfamily},
  numbers=none,
  numberstyle=\tiny\color{gray},
  keywordstyle=\color{blue},
  commentstyle=\color{dkgreen},
  stringstyle=\color{mauve},
  breaklines=true,
  breakatwhitespace=true,
  tabsize=3
}

%Gummi|065|=)
\title{\textbf{Lab 4}}
\author{Lucie Kaffee\\
		Charlie Kritschmar
		}
\date{}

\begin{document}

\maketitle

\section{$x^{y}$}
\begin{equation}
x^{16} = (((x^ {2})^{2})^{2})
\end{equation}

Therefore 4 multiplications are necessary to calculate $x^{16}$.

\begin{equation}
x^{21} = x^ {20} \cdot x = x^{16} \cdot x^{4} \cdot x
\end{equation}

In this case, 6 multiplications are necessary since $x^{4}$ is calculated on the way to $x^{16}$. 

\begin{equation}
x^{341} = x^ {340} \cdot x = x^{256} \cdot x^{64} \cdot x^{16} \cdot x^{4} \cdot x
\end{equation}

To calculate $x^{256}$ eight multiplications necessary, for $x^{64}$ six, for $x^{16}$ four and for $x^{4}$ two. Sice we will calculate $x^{64}$, $x^{16}$ and $x^{4}$ on the way to $x^{256}$, nine multiplications are necessary in this case.

Equially calculated as the ones before, we were able to see that for the last one it's 17 multiplications.

For the algorithm we used \href{https://de.wikipedia.org/wiki/Bin%C3%A4re_Exponentiation#Pseudocode_.28Algorithmus.29}{Wikipedia Bin\"are Exponentation} as a resource.

We converted the $y$ to its binary representation and iterated over that. Depending on whether the next number (or in this case char) was a zero or a one, we either squared and multiplicated with $x$ or just squared the result.  

To see how many calculations were needed in our program, we use a counter called numberCalculation.

\section{}
\begin{equation}
\begin{split}
7^{x} \equiv 10 \mod 31 \\
7^2 \equiv 18 \\
7^3 \equiv 18 \cdot 7 \equiv 2 \\
7^4 \equiv 2 \cdot 7 \equiv 14 \\
7^5 \equiv 14 \cdot 7 \equiv 5 \\
7^6 \equiv 5 \cdot 7 \equiv 4 \\
7^7 \equiv 4 \cdot 7 \equiv 28 \\
7^8 \equiv 28 \cdot 7 \equiv 196 \equiv 10 \mod 31 \\
\end{split}
\end{equation}


\section{Code}
\subsection{$x^{y}$}
\lstinputlisting[language=Java]{Main.java}
\lstinputlisting[language=Java]{TaskOne.java}
\subsection{}
\lstinputlisting[language=Java]{Main2.java}
\lstinputlisting[language=Java]{Lab04.java}
\end{document}
